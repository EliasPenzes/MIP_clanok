% Metódy inžinierskej práce

\documentclass[10pt,twoside,slovak,a4paper]{article}

\usepackage[slovak]{babel}
\usepackage[IL2]{fontenc} 
\usepackage[utf8]{inputenc}
\usepackage{graphicx}
\usepackage{url} 
\usepackage{hyperref} 

\usepackage{cite}

\pagestyle{headings}

\title{Time arrival prediction\thanks{Semestrálny projekt v predmete Metódy inžinierskej práce, ak. rok 2021/22, vedenie: Ing. Bystrík Bindas}}

\author{Eliáš Pénzeš\\[2pt]
	{\small Slovenská technická univerzita v Bratislave}\\
	{\small Fakulta informatiky a informačných technológií}\\
	{\small \texttt{xpenzese@stuba.sk}}
	}


\date{\small 19. september 2021} 
\begin{document}

\maketitle

\begin{abstract}
Predpoveď času príchodu je významnou súčasťou verejnej dopravy. Existuje viacero spôsobov akými môžeme tento čas predpovedať. Presnosť a spoľahlivosť jednotlivých modelov predpovede je kľúčovou zložkou. V mojej práci by som sa rád zameral na rôzne modely a možnosti predpovede príchodu prostriedkov verejnej dopravy a na to aké modely sú v dnešnej dobe najčastejšie používané. Taktiež by som rád upresnil základy a princípy na akých jednotlivé modely fungujú. Na ich presnosť a možnosť adaptovať sa na rušnosť dopravy v reálnom čase alebo rôzne neočakávané situácie. Tiež by som rád spomenul aj to, ako spoľahlivé predpovede príchodu ovplyvňujú život ľudom ktorý cestujú verejnou dopravou. 
\end{abstract}
\section{Úvod}

\section{Model 1} \label{nejaka}

\section{Dôležitá časť} \label{dolezita}

\section{Ešte dôležitejšia časť} \label{dolezitejsia}

\section{Záver} \label{zaver}

\end{document}